\documentclass{beamer}
\usepackage{ctex}  % comment this line to use pure English
\usepackage{minted}
% \usepackage{listings}
\usepackage[utf8]{inputenc}

% \setminted{xleftmargin=1cm}  % for code line number 

% \useoutertheme{split}
\usetheme{Madrid}
\usecolortheme{default}
\setbeamertemplate{navigation symbols}{}

\makeatother
\setbeamertemplate{footline}
{
  \leavevmode%
  \hbox{%
  \begin{beamercolorbox}[wd=.4\paperwidth,ht=2.25ex,dp=1ex,center]{author in head/foot}%
    \usebeamerfont{author in head/foot}\insertshortauthor
  \end{beamercolorbox}%
  \begin{beamercolorbox}[wd=.6\paperwidth,ht=2.25ex,dp=1ex,center]{title in head/foot}%
    \usebeamerfont{title in head/foot}\insertshorttitle\hspace*{3em}
    \insertframenumber{} / \inserttotalframenumber\hspace*{1ex}
  \end{beamercolorbox}}%
  \vskip0pt%
}
\makeatletter
\setbeamertemplate{navigation symbols}{}


%------------------------------------------------------------
%This block of code defines the information to appear in the
%Title page
\title[Intro to jax] %optional
{A simple introduction to JAX}

\subtitle{A quick view}

\author[Feng Wang] % (optional)
{Feng Wang\inst{1}}

\institute[megvii] % (optional)
{
  \inst{1}%
  Megvii\\
  wangfeng02@megvii.com
}

\date[megvii 2022] % (optional)
{Megvii, December 2022}

% \logo{\includegraphics[height=1cm]{overleaf-logo}}

%End of title page configuration block
%------------------------------------------------------------



%------------------------------------------------------------
%The next block of commands puts the table of contents at the 
%beginning of each section and highlights the current section:

\AtBeginSection[]
{
  \begin{frame}
    \frametitle{Table of Contents}
    \tableofcontents[currentsection]
  \end{frame}
}
%------------------------------------------------------------


\begin{document}

%The next statement creates the title page.
\frame{\titlepage}


%---------------------------------------------------------
%This block of code is for the table of contents after
%the title page
\begin{frame}
\frametitle{Table of Contents}
\tableofcontents
\end{frame}
%---------------------------------------------------------


\section{Jax quick start}

%---------------------------------------------------------
\begin{frame}[fragile]
\frametitle{Jax quick start}
Jax 的核心理念是函数式编程,更关心数据是如何映射的,而命令式编程关心解决问题的步骤。\newline

举个例子,考虑反转二叉树的面试题目,常规python代码是(命令式编程):

\begin{minted}{python}
def invertTree(root):
    if root is None:
        return None

    root.left, root.right = invertTree(root.right), \
        invertTree(root.left)
    return root

\end{minted}

含义很简单:首先判断节点是否为空,然后翻转左树和右树,最后左右互换。
\end{frame}

\begin{frame}[fragile]
\frametitle{Jax quick start}

函数式编程则是另一个视角:所谓“翻转二叉树”,可以看做是要得到一颗和原来二叉树对称的新二叉树。
这颗新二叉树的特点是每一个节点都递归地和原树相反。\newline

函数编程则可能这样写:\newline

\begin{minted}{python}
def invert(node):
    if node is None:
        return None
    else
        return Tree(node.value, \
            invert(node.right), invert(node.left))

\end{minted}

这段代码体现的思维,就是旧树到新树的映射——对一颗二叉树而言,它的镜像树就是左右节点递归镜像的树。
\end{frame}

\begin{frame}[fragile]
\frametitle{Jax quick start}

函数式编程思想在Jax中的一些体现:

\begin{itemize}
    \item<1-> inplace概念\newline
    jax中没有inplace操作,所有的操作都是通过map逻辑映射出来,也就是:
    \begin{minted}{python}
    x = jnp.sigmoid(jnp.reshape(x, (1, -1)))
    \end{minted}
    \item<2-> model状态\newline
    model也是没有状态的,也没有所谓的初始化,输入决定了模型的参数形状。参数决定了模型forward结果。
    \begin{minted}{python}
    param = model.init(rng, model_input)
    output = model.apply(param, inputs, training=True)
    \end{minted}
    \item<3-> grad和vmap\newline
    梯度就是针对function的一个操作,向量化就是单个输入的映射concat出来的。
    \begin{minted}{python}
    grad_fn = jax.grad(model_fn)
    batch_fn = jax.vmap(model_fn)
    \end{minted}
\end{itemize}
\end{frame}


\begin{frame}[fragile]
\frametitle{Jax quick start}

使用Jax很简单:想象你有一个Jax Array,使用numpy中的API来操作这个Array。\newline
Jax没有device这样的认知负担,当你安装了GPU version的Jaxlib,默认就是GPU,在没有检测到GPU的时候,就是CPU。

\begin{minted}{python}
import jax.numpy as jnp
from jax import random

key = random.PRNGKey(0)
size = 3000
x = random.normal(key, (size, size), dtype=jnp.float32)
%timeit jnp.dot(x, x.T).block_until_ready()  # GPU
\end{minted}

\begin{block}{NOTE}
这里 block\_until\_ready方法是必要的,\textbf{在测速的时候一定要加上}。
\end{block}

\end{frame}

\begin{frame}[fragile]
\frametitle{Jax quick start}

Jax的核心加速方法就是jit。相比原始的python code,jit可以获得十倍之上的加速。\newline
\begin{minted}[linenos]{python}
def selu(x, alpha=1.67, lmbda=1.05):
  return lmbda * jnp.where(x > 0, x, alpha * jnp.exp(x) - alpha)

x = random.normal(key, (1000000,))
%timeit selu(x).block_until_ready()

selu_jit = jit(selu)
%timeit selu_jit(x).block_until_ready()
\end{minted}

\begin{minted}{shell-session}
1.07 ms ± 261 µs per loop (mean ± std. dev. of 7 runs)
127 µs ± 1.43 µs per loop (mean ± std. dev. of 7 runs)
\end{minted}

\end{frame}

\begin{frame}[fragile]
\frametitle{Jax quick start}

习惯了写torch,刚开始写Jax可能不舒服的地方:\newline

\begin{itemize}
    \item<1-> 不能inplace update\newline
    前面说过,jax中没有inplace操作,所以涉及到array的更新也不能直接在array上进行操作:
    \begin{minted}{python}
    # Numpy
    np_array = np.zeros((3,3), dtype=np.float32)
    np_array[1, :] = 1.0 # In place, mutating update
    # Jax
    jax_array = jnp.zeros((3,3), dtype=jnp.float32)
    updated_array = jax_array.at[1, :].set(1.0)
    new_jax_array = jax_array.at[::2, 3:].add(7.)
    \end{minted}
\end{itemize}

\end{frame}


\begin{frame}[fragile]
\frametitle{Jax quick start}

习惯了写torch,刚开始写Jax可能不舒服的地方:\newline

\begin{itemize}
    \item<1-> 随机数需要自己造\newline
    jax里面所有生成随机数的方法都需要一个显示传入的key
    \begin{minted}{python}
    from jax import random
    key = random.PRNGKey(0)
    print(random.normal(key, shape=(1,)))
    key, subkey = random.split(key)
    key, *subkeys = random.split(key, 4)  # multi subkey
    for subkey in subkeys:
        print(random.normal(subkey, shape=(1,)))
    \end{minted}
\end{itemize}

\end{frame}
%---------------------------------------------------------

%---------------------- To JIT or not to JIT ----------------

\section{To JIT or not to JIT}

\begin{frame}[fragile]
\frametitle{To JIT or not to JIT}

虽然Jax的jit真的很香,但是jit也是需要在限制下才能满足的,一些操作并不能和jit一起使用。\newline
为了方便查看jit之后的函数形式,jax提供了make\_jaxpr来查看jit之后的函数形式。
\begin{minted}{python}
from jax import random, make_jaxpr

def f(x, y):
  return jnp.dot(x + 1, y + 1)

x = np.random.randn(3, 4)
y = np.random.randn(4)
make_jaxpr(f)(x, y)
\end{minted}
\end{frame}

\begin{frame}[fragile]
\frametitle{To JIT or not to JIT}

jaxpr(Jax print)输出的jit之后的函数形式:
\begin{minted}{python}
>>> make_jaxpr(f)(x, y)
{ lambda  ; a b.
  let c = add a 1.0
      d = add b 1.0
      e = dot_general[
        dimension_numbers=(((1,), (0,)), ((), ()))
        precision=None ] c d
  in (e,) }
\end{minted}
\end{frame}


\begin{frame}[fragile]
\frametitle{To JIT or not to JIT}

因为jit是在没有数组内容信息的情况下完成的,所以函数中的控制流语句不能依赖于traced value。\newline
\begin{minted}{python}
@jit
def f(x, neg):
  return -x if neg else x

f(1, True)  # crash here
\end{minted}

\begin{block}{NOTE}
想要正常trace,需要声明neg为非trace结果(static参数)。
此时,当neg发生变化的时候,jit会重新编译函数f。
\end{block}

\begin{minted}{python}
@functools.partial(jit, static_argnums=(1,))
def f(x, neg):
  return -x if neg else x
\end{minted}

\end{frame}

% ---------------------- page ----------------------
\begin{frame}[fragile]
\frametitle{To JIT or not to JIT}

JIT会在第一次运行的时刻编译函数,并且将编译好的函数做cache,所以如果使用了一些全局变量,
会有一些side-effect。\newline

\begin{block}{NOTE}
被jit装饰的函数会在第一次运行的时候进行编译缓存,所以jit函数的第一次运行会比较耗时。
\end{block}

\begin{minted}{python}
y = 0

# @jit   # Different behavior with jit
def impure_func(x):
  print("Inside:", y)
  return x + y

for y in range(3):
  print("Result:", impure_func(y))
\end{minted}
\end{frame}

\begin{frame}[fragile]
\frametitle{To JIT or not to JIT}

运行结果(纯函数版本和jit版本):

\begin{columns}
\column{0.5\textwidth}
\begin{minted}{python}
    # pure func
    Inside: 0
    Result: 0
    Inside: 1
    Result: 2
    Inside: 2
    Result: 4
\end{minted}

\column{0.5\textwidth}
\begin{minted}{python}
# jit func
Inside: 0
Result: 0
Result: 1
Result: 2
\end{minted}

\end{columns}

\end{frame}

% ---------------------- page ----------------------
\begin{frame}[fragile]
jit对函数进行编译的时候会做function的fuse,一些不必要的操作也会做去除,所以执行结果不同也是有可能的。
\begin{minted}{python}
def f(x):
    return jnp.log(jnp.exp(x))
x = 100.0
>>> print(f(x))
inf
>>> print(jit(f)(x))
100.0
\end{minted}
可以看出在上边的例子中log和exp在fuse之后就互相抵消掉了,函数返回的原始结果也不同了。\newline
关于fuse的规则jax官方文档中也没有详细阐述,所以关于这部分的细节只能看code。

\end{frame}


\begin{frame}[fragile]
\frametitle{To JIT or not to JIT}

来看下面一个简单的class,想想调用是否会导致crash,如何fix?
\begin{examples}[bug fix]
\begin{minted}{python}
class CustomClass:
  def __init__(self, x: jnp.ndarray, mul: bool):
    self.x = x
    self.mul = mul

  @jit  # <---- How to do this correctly?
  def calc(self, y):
    if self.mul:
      return self.x * y
    return y

c = CustomClass(2, True)
c.calc(3) 
\end{minted}
\end{examples}

\end{frame}
%---------------------------------------------------------


%---------------------- Flax ----------------

\section{Flax \& Optax: neural network library}

\begin{frame}[fragile]
\frametitle{Flax \& Optax: neural network library}
用flax写成的一个简单例子:
\begin{minted}{python}
from flax import linen as nn 

class CNN(nn.Module):
  @nn.compact
  def __call__(self, x):
    x = nn.Conv(features=32, kernel_size=(3, 3))(x)
    x = nn.relu(x)
    x = nn.avg_pool(x, window_shape=(2, 2), strides=(2, 2))
    x = x.reshape((x.shape[0], -1))  # flatten
    x = nn.Dense(features=10)(x)
    return x

model = CNN()
x = jnp.empty((4, 28, 28, 1))
params = model.init(PRNGKey(42), x)
y = model.apply(params, x)

\end{minted}
\end{frame}

\begin{frame}[fragile]
\frametitle{Flax \& Optax: neural network library}
模型训练参数的创建:
\begin{minted}{python}
from flax.training import train_state
import optax

def create_train_state(rng, learning_rate, momentum):
  cnn = CNN()
  params = cnn.init(rng, jnp.ones([1, 28, 28, 1]))['params']
  return train_state.TrainState.create(
      apply_fn=cnn.apply,
      params=params,
      tx=optax.sgd(learning_rate, momentum)
  )
\end{minted}
\end{frame}

\begin{frame}[fragile]
\frametitle{Flax \& Optax: neural network library}
模型的单步训练:
\begin{minted}{python}
@jax.jit
def train_step(state, batch):
  image, label = batch['image'], batch['label']

  def loss_fn(params):
    logits = CNN().apply({'params': params}, image)
    loss = cross_entropy_loss(logits=logits, labels=label)
    return loss, logits

  grad_fn = jax.grad(loss_fn, has_aux=True)
  grads, logits = grad_fn(state.params)
  state = state.apply_gradients(grads=grads)
  metrics = compute_metrics(logits=logits, labels=label)
  return state, metrics
\end{minted}
\end{frame}

\begin{frame}[fragile]
\frametitle{Flax \& Optax: neural network library}
框架下的训练逻辑:
\begin{minted}{python}
for data in dataloader:
    state, metrics = train_step(state, data)
print(metrics)
\end{minted}
最后模型返回的state.params就是训练好的模型参数,保存即可。
\begin{minted}{python}
from flax.serialization import msgpack_serialize

def save_params(params, file):
    with open(file, "wb+") as f:
        serialized = msgpack_serialize(params.unfreeze())
        f.write(serialized)
    print(f"Saved successfully to {file}")

save_params(state.params, "params.npy")
\end{minted}
\end{frame}

\begin{frame}{}
  \centering \Large
  \emph{Thank you!}
  \emph{Any questions or comments are welcomed!}
\end{frame}
%---------------------------------------------------------
%Highlighting text
% \begin{frame}
% \frametitle{Sample frame title}

% In this slide, some important text will be
% \alert{highlighted} because it's important.
% Please, don't abuse it.

% \begin{block}{Remark}
% Sample text
% \end{block}

% \begin{alertblock}{Important theorem}
% Sample text in red box
% \end{alertblock}

% \begin{examples}
% Sample text in green box. The title of the block is ``Examples".
% \end{examples}
% \end{frame}
%---------------------------------------------------------

%Example of the \pause command
% \begin{frame}
% In this slide \pause
% the text will be partially visible \pause
% And finally everything will be there
% \end{frame}

%---------------------------------------------------------
%Two columns
% \begin{frame}
% \frametitle{Two-column slide}

% \begin{columns}

% \column{0.5\textwidth}
% This is a text in first column.
% $$E=mc^2$$
% \begin{itemize}
% \item First item
% \item Second item
% \end{itemize}

% \column{0.5\textwidth}
% This text will be in the second column
% and on a second tought this is a nice looking
% layout in some cases.
% \end{columns}
% \end{frame}
%---------------------------------------------------------

\end{document}